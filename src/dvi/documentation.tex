\documentclass[12pt,oneside,a4paper]{article}
\usepackage[utf8]{inputenc}
\usepackage[T1]{fontenc}
\usepackage{graphicx}
\usepackage{enumitem}
\usepackage{enumitem}
\usepackage{hyperref}

\begin{document}

\title{Документация для проекта BrickGame v1.0 aka Tetris}
\author{mylendad}
\date{Март 2025}
\maketitle

\section{Введение}

Проект BrickGame v1.0 aka Tetris представляет собой реализацию классической аркадной игры Тетрис с использованием языка программирования C и библиотеки ncurses для терминального интерфейса.

\section{Структура проекта}

\subsection{Библиотека Tetris (\texttt{src/brick\_game/tetris})}

\begin{itemize}[label=--]
    \item Файлы с исходным кодом, реализующие логику игры Tetris, включая обработку ввода команд управлениея пользавателем, а так же реализацию конечного автомата для формализации логики игры.
    \item Файлы с исходным кодом, реализующие логику отображения фигур с игровым полем, а так же табло счета очков, уровней, и показа следующей фигуры.
\end{itemize}

\subsection{Терминальный интерфейс (\texttt{src/gui/cli})}

\begin{itemize}[label=--]
    \item Файлы с исходным кодом, отвечающие за визуализацию игры в терминале с использованием библиотеки ncurses.
    \item Реализация отрисовки игрового поля, управления вводом пользователя и отображения текущего состояния игры.
\end{itemize}

\section{Сборка проекта}

Проект использует систему сборки \texttt{make} с Makefile, включающим следующие цели:

\begin{itemize}
    \item \texttt{all}: Очистка временных файлов и папок, сборка проекта, запуск тестов, запуск gcov - репорта.
    \item \texttt{install}: Установка программы в систему.
    \item \texttt{uninstall}: Удаление программы из системы.
    \item \texttt{clean}: Очистка временных файлов и папок.
    \item \texttt{dvi}: Создание файла DVI.
    \item \texttt{dist}: Создание архива, содержащего необходимые файлы для сборки и использования программы.
\end{itemize}

\section{Требования к среде выполнения}

Проект предполагает использование языка программирования C11, компилятора gcc и библиотеки ncurses для терминального интерфейса.

\section{Инструкции по установке и запуску}

\begin{enumerate}
    \item \textbf{Установка зависимостей:}
        \begin{itemize}
            \item Убедитесь, что у вас установлен компилятор gcc.
            \item Установите библиотеку ncurses.
        \end{itemize}
    \item \textbf{Сборка проекта:}
        \begin{itemize}
            \item Выполните \texttt{make all} для сборки, запуска тестов, запуска gcov - репорта проекта.
        \end{itemize}
    \item \textbf{Установка:}
        \begin{itemize}
            \item Выполните \texttt{make install} для установки программы в систему.
        \end{itemize}
    \item \textbf{Запуск:}
        \begin{itemize}
            \item Выполните \texttt{make run} для запуска программы.
        \end{itemize}
\end{enumerate}

\section{Использование программы}

\begin{enumerate}
    \item \textbf{Управление:}
        \begin{itemize}
            \item Используйте стрелки влево и вправо для перемешения фигуры по горизонтали.
            \item Нажмите клавишу вниз для падения фигуры.
            \item Для поворота фигуры используйте клавишу пробела.
            \item Чтобы поставить игру на паузу используйте 'p.
            \item Для снятия игры с паузы повторно нажмите 'p'.
            \item Для выхода из игры используйте 'q'.
        \end{itemize}
    \item \textbf{Механики игры:}
        \begin{itemize}
            \item Вращение фигур. 
            \item Перемещение фигур вправо и влево.
            \item Падение фигуры.
            \item Механика начисления очков за уничтожение заполненных линий.
            \item Начисление очков будет происходить следующим образом:

                    - 1 линия — 100 очков;
                    - 2 линии — 300 очков;
                    - 3 линии — 700 очков;
                    - 4 линии — 1500 очков.

            \item Механика уровней: каждый раз, когда игрок набирает 600 очков, уровень увеличивается на 1. Повышение уровня увеличивает скорость движения фигур. Максимальное количество уровней — 10.
            \item Хранение максимального количества очков.
            \item Меню паузы.

        \end{itemize}
    \item \textbf{Завершение игры:}
        \begin{itemize}
            \item Игра завершается, когда достигнута верхняя граница игрового поля.
        \end{itemize}
\end{enumerate}

\section{Тестирование}

Проект включает в себя unit-тесты с использованием библиотеки check. Тестируются функции логики игры и функции отрисовки. Покрытие библиотеки тестами составляет более 80\%.

\end{document}